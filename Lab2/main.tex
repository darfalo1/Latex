\documentclass[journal]{IEEEtran}
\IEEEoverridecommandlockouts
%%%%%%%%%%%%%%%%%%%%%%%%%%%%%%%%%%%%%%
%%%%%%%% PRINCIPALES PAQUETES %%%%%%%%
%%%%%%%%%%%%%%%%%%%%%%%%%%%%%%%%%%%%%%
%\usepackage{fancyhdr}
\usepackage{circuitikz}
\usepackage{graphicx}
\usepackage{float}
\usepackage[utf8]{inputenc}
\usepackage{comment}
\usepackage{amsmath}
\usepackage[spanish,es-tabla]{babel}
\usepackage{amssymb}
%\usepackage{pgfplots}
%\usepackage{color}
%\usepackage{hyperref}
%\usepackage{wrapfig}
%\usepackage{array}
\usepackage{multirow}
%\usepackage{adjustbox}
%\usepackage{nccmath}
%\usepackage{subfigure}
%\usepackage{amsfonts,latexsym} % para tener disponibilidad de diversos simbolos
%\usepackage{enumerate}
%\usepackage{booktabs}

%\usepackage{threeparttable}
\usepackage{array,colortbl}
%\usepackage{ifpdf}
%\usepackage{rotating}
%\usepackage{cite}
%\usepackage{stfloats}
%\usepackage{url}
%\usepackage{listings}
\usepackage[spanish]{babel}
%%%%%%%%%%%%%%%%%%%%%%%%%%%%%%%%%%%%%%%%%%%
%%% CREAR Y REESCRIBIR ALGUNOS COMANDOS %%%
%%%%%%%%%%%%%%%%%%%%%%%%%%%%%%%%%%%%%%%%%%%
\newcolumntype{P}[1]{>{\centering\arraybackslash}p{#1}}  %% Se crea un nuevo tipo de columna llamada P.
\newcommand{\tabitem}{~~\llap{\textbullet}~~}
\newcommand{\ctt}{\centering\scriptsize\textbf} %%\ctt abrevia el comando \centering\scriptsize\textbf
\newcommand{\dtt}{\scriptsize\textbf} %%\dtt abrevia el comando \scriptsize\textbf
\renewcommand\IEEEkeywordsname{Palabras clave}
%%%%%%%%%%%%%%%%%%%%%%%%%%%%%%%%%%%%%%%%%%%


% correct bad hyphenation here
\hyphenation{op-tical net-works semi-conduc-tor} %% Con este comando se especifican como pueden seprarse las sílabas adecuadamente en caso una palabra quede en dos lineas diferentes de texto

\graphicspath{ {figs/} }  %%Ruta donde se encuentran las imágenes, que esté vacio indica que las imagenes están dentro de la misma carpeta que contiene el archivo .tex


%%%%%%%%%%%%%%%%%%%%%%%%%%%%%%%%%%%%%%%%%%%%%%%%%%%%%%%%%%
%%% ENCABEZADO DE LAS PÁGINAS TIPO UNIVERSIDAD CENTRAL %%%
%%%%%%%%%%%%%%%%%%%%%%%%%%%%%%%%%%%%%%%%%%%%%%%%%%%%%%%%%%
\newcommand{\MYhead}{\smash{\scriptsize
\hfil\parbox[t][\height][t]{\textwidth}{\centering
\begin{picture}(0,0) \put(-0,-15){\includegraphics[scale=0.08 ]{Fig/Logo_Escudo_Vertical.png}} \end{picture} \hspace{6.4cm}
Laboratorio Electronica I  \hspace{5.15cm} Versión 1.1\\
\hspace{6.2cm} FACULTAD DE INGENIERÍA \hspace{4.45cm} Periodo 2025-1\\
\underline{\hspace{ \textwidth}}}\hfil\hbox{}}}
\makeatletter
% normal pages
\def\ps@headings{%
\def\@oddhead{\MYhead}%
\def\@evenhead{\MYhead}}%
% title page
\def\ps@IEEEtitlepagestyle{%
\def\@oddhead{\MYhead}%
\def\@evenhead{\MYhead}}%
\makeatother
% make changes take effect
\pagestyle{headings}
% adjust as needed
\addtolength{\footskip}{0\baselineskip}
\addtolength{\textheight}{-1\baselineskip}
%%%%%%%%%%%%%%%%%%%%%%%%%%%%%%%%%%%%%%%%%%%%%%%%%%%%%%%%%%
%%%%%%%%%%%%%%%%%%%%%%%%%%%%%%%%
%%%%% INICIO DEL DOCUMENTO %%%%%
%%%%%%%%%%%%%%%%%%%%%%%%%%%%%%%%
\begin{document}
%%%%%%%%%%%%%%%%%%%%%%%%%%%%
%%% TÍTULO DEL DOCUMENTO %%%
%%%%%%%%%%%%%%%%%%%%%%%%%%%%
\title{Aplicación de transistores BJT (NE555).}
%%%%%%%%%%%%%%%%%%%%%%%%%%%%
%%%%%%%%% AUTORES %%%%%%%%%
%%%%%%%%%%%%%%%%%%%%%%%%%%%
\author{Fabián López Pérez (Código: 20232005069).} 

% Comando que indica la generación del título
\maketitle
%%%%%%%%%%%%%%%%%%%%%
%%%%%% RESUMEN %%%%%%
%%%%%%%%%%%%%%%%%%%%%
\begin{abstract}
.........................................
\end{abstract}

% En el resumen no se recomienda colocar citaciones bibliográficas.
%%%%%%%%%%%%%%%%%%%%%%
%%% PALABRAS CLAVE %%%
%%%%%%%%%%%%%%%%%%%%%%
\begin{IEEEkeywords}
Corriente, Tensión, Capacitancia, Inductancia, análisis de circuitos, inductores, capacitores.
\end{IEEEkeywords}
%%%%%%%%%%%%%%%%%%%%%%
%\IEEEpeerreviewmaketitle
%%%%%%%%%%%%%%%%%%%%%%%%%%%%%%%%%%%%%
%%% PRIMERA SECCIÓN DEL DOCUMENTO %%%
%%%%%%%%%%%%%%%%%%%%%%%%%%%%%%%%%%%%%

\section{\textbf{Introducción.}}
\IEEEPARstart{E}{} ..........................
\section{\textbf{Objetivo.}}
    \begin{itemize}
        \item ....................................
    \end{itemize}

\section{\textbf{Marco teórico.}}
           
\section{\textbf{Requerimientos.}}
    \vspace{0.2cm}
    \subsection{\textbf{Elementos Funcionales necesarios.}}
        \vspace{0.2cm}
        \begin{itemize}
            \item Protoboard.
            \item IC NE555.
            \item Capacitores.
            \item Potenciómetros.
            \item Resistores.
            \item LEDS
        \end{itemize}

\section{\textbf{Procedimiento.}}
    \subsection{Montaje Circuito.}    

   
        \begin{figure}[ht]
            \centering
            \includegraphics[width=1\linewidth]{Fig/Montaje 1.PNG}
            \caption{Montaje 1.}
            \label{montaje1.}
        \end{figure}
    
    
    
\section{\textbf{Resultados.}}

    \begin{table}[ht]
                \begin{center}
                    \begin{tabular}{ |c|c|c|c| } 
                        \hline
                        $V_{c1}$&$V_{c2}$ &$V_{in}$ &$V_{out}$ \\
                        \hline
                        $15.57v$& $30.1v$ & $15.57vp$ & $30.1v $\\ 
                        \hline
                    \end{tabular}
                \end{center} 
                \caption{Valores del Montaje 1(Duplicador).}
                \label{tabla:1}
        \end{table}
        
\section{\textbf{Análisis de resultados.}}
    \begin{itemize}
        \item 
    \end{itemize}
\section{\textbf{Conclusiones.}}

%%%%%%%%%%%%%%%%%%%%%%%%%%
%%%%%% BIBLIOGRAFIA %%%%%%
%%%%%%%%%%%%%%%%%%%%%%%%%%
\bibliographystyle{IEEEtran}




%%%%%%%%%%%%%%%%%%%%%%%%%%


\end{document}
%%%%%%%%%%%%%%%%%%%%%%%%%%%%%%%%
%%%%%% FIN DEL DOCUMENTO %%%%%%%
%%%%%%%%%%%%%%%%%%%%%%%%%%%%%%%%
